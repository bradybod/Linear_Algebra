\documentclass[]{exam}

\usepackage{amsmath, amssymb}
\usepackage[margin=1in]{geometry}
\usepackage[en-US, showdow]{datetime2}
\usepackage{graphicx}
\usepackage{multicol,enumitem}
\usepackage{tikz}

\newenvironment{smallbmatrix}
{\left[\begin{smallmatrix}}
{\end{smallmatrix}\right]}
\newcommand*\circled[1]{\tikz[baseline=(char.base)]{
            \node[shape=circle,draw,inner sep=2pt] (char) {#1};}}

\title{Homework 5\\
	Due \DTMdate{2020-03-31} %Add a % at the beginning of the line to remove the due date from the title (or just delete this line entirely) 
	}
\date{ %\today %You can remove the first % in this line to show the current days date.
	}
\author{Name:Brady Bodily \\
		\footnotesize Collaborators: Mitch Pound, Danny Clyde, Jaxon Willard  \\
		\footnotesize Recitation Section: \circled{504} /505/506
		%You should probably change this to your actual name, at least if you want credit.
		}
\begin{document}
\maketitle

\printanswers %comment out this line to hide your answers.

\begin{questions}
	\question Construct a matrix with the required properties or say why it is impossible. If it is impossible to construct a matrix with the given property you must give an explanation that relies on orthogonality of subspaces.:
	
	\begin{parts}
		\part Construct a matrix whose column space contains 
		$\begin{bmatrix}1 \\2\\-3\end{bmatrix}$ and 
		$\begin{bmatrix}2 \\-3\\5\end{bmatrix}$ and
		whose nullspace contains 
		$\begin{bmatrix}1 \\1\\1\end{bmatrix}$.
		
		\begin{solution}
				Using the given vectors from the column space and the null space, we can create the matrix $A$=
				$\begin{bmatrix}
				1 & 2 & -3 \\
				2 & -3 & 1 \\
				-3 & 5 & -2 \\ 
				\end{bmatrix}$.
				The third column is created by adding the first two columns, then setting the third column's value equal to the additive inverse of the total of columns 1 and 2.  Doing this will give the $\vec{0}$ when the dot product is used on $A\vec{x}=\vec{0}$

		\end{solution}
			
		\part Construct a matrix whose row space contains 
		$\begin{bmatrix}1 \\2\\-3\end{bmatrix}$ and 
		$\begin{bmatrix}2 \\-3\\5\end{bmatrix}$ and
		whose nullspace contains 
		$\begin{bmatrix}1 \\1\\1\end{bmatrix}$.
		
		\begin{solution}
							Using the vectors of the rowspace we can construct the matrix $A= \begin{bmatrix}
			    1 & 2 & -3 \\
			    2 & -3 & 5 \\
				\end{bmatrix}$.
				Using the fundemental therom of linear algebra, the row space adn null space must be orthogonal.  To determine orthogonality the row space dotted with the null space will return a result of the $\vec{0}$ when true.  If the result is a non zero vector they are not orthogonal. So, 
				\newline
				$\begin{bmatrix}
			    1 & 2 & -3 \\
			    2 & -3 & 5 \\
				\end{bmatrix}$
				$\begin{bmatrix}
			    1 \\
			    1 \\
			    1 \\
				\end{bmatrix} \neq \vec{0}$, therefore it is not orthogonal.

		\end{solution}
		
		\part Construct a matrix $A$ such that
		$A\vec{x} = \begin{bmatrix}1 \\1\\1\end{bmatrix}$
		has a solution and satisfies 
		$A^T \begin{bmatrix}1 \\0\\0\end{bmatrix}
		= \begin{bmatrix}0 \\0\\0\end{bmatrix}$
		
		\begin{solution}
							Per the fundamental theorem of linear algebra $C(A^T) \perp N(A)$, for $A\vec{x} = \begin{bmatrix}
				    1 \\
				    1 \\
				    1 \\
				\end{bmatrix}$.
				The vector 
				$\vec{x} = \begin{bmatrix}
				    1 \\
				    1 \\
				    1 \\
				\end{bmatrix}$
				must be in the column space of A.  For 
				$A^T \begin{bmatrix}
				    1 \\
				    0 \\
				    0 \\
				\end{bmatrix} = 
				\begin{bmatrix}
				    0 \\
				    0 \\
				    0 \\
				\end{bmatrix} $ then the vector
				$\begin{bmatrix}
				    1 \\
				    0 \\
				    0 \\
				\end{bmatrix}$ must be in the null space of $A^T$. To determine if they are orthogonal we will take the dot product of $\begin{bmatrix}
				    1 \\
				    1 \\
				    1 \\
				\end{bmatrix}\begin{bmatrix}
                    1 & 0 & 0
				\end{bmatrix} =  1$ which is not 0, therefore is not orthogonal and not possible.
		\end{solution}
		
		\part Construct a matrix $A$ which is a nonzero matrix 
		and whose rows are each orthogonal to every column.
		
		\begin{solution}
            	The $\vec{0}$ is orthogonal to itself and every other vector, given the matrix $\begin{bmatrix}
				    0 & 1 \\
				    0 & 0 \\
				\end{bmatrix}$. We can set up the following tests for orthogonal tests.
				\newline
				$\begin{bmatrix}
				    0 & 1
				\end{bmatrix}\begin{bmatrix}
                    1 \\
                    0
				\end{bmatrix} =  0$
				\newline
				$\begin{bmatrix}
				    0 & 0
				\end{bmatrix}\begin{bmatrix}
                    1 \\
                    0
				\end{bmatrix} =  0$
				\newline
				$\begin{bmatrix}
				    0 & 1
				\end{bmatrix}\begin{bmatrix}
                    0 \\
                    0
				\end{bmatrix} =  0$
				\newline
				$\begin{bmatrix}
				    0 & 0
				\end{bmatrix}\begin{bmatrix}
                    0 \\
                    0
				\end{bmatrix} =  0$.
				\newline
				As they are all equal to 0 we have found a non zero matrix with every row orthogonal to every column.

		\end{solution}
		
		\part Construct a matrix whose columns add up to $\vec{0}$
		and whose rows add to a row of $1$'s.
		
		\begin{solution}
							Given a $3\times3$ matrix 
				$A=\begin{bmatrix}
				    a_1 & a_2 & a_3 \\
				    a_4 & a_5 & a_6 \\
				    a_7 & a_8 & a_9 \\
				\end{bmatrix}$
				by the column method of matrix multiplication for $\begin{bmatrix}
				    a_1 & a_2 & a_3 \\
				    a_4 & a_5 & a_6 \\
				    a_7 & a_8 & a_9 \\
				\end{bmatrix}\begin{bmatrix}
				    1 \\
				    1 \\
				    1 \\
				\end{bmatrix}=
				\begin{bmatrix}
				    0 \\
				    0 \\
				    0 \\
				\end{bmatrix}$
			    the matrix $
			    \begin{bmatrix}
				    1 \\
				    1 \\
				    1 \\
				\end{bmatrix}$
				must be in the null space of the matrix $A$ in order to be true.  Then by using the row method of matrix multiplication $\vec{u}A=\begin{bmatrix}
				    1 & 1 & 1
				\end{bmatrix}$, the vector $\begin{bmatrix}
				    1 & 1 & 1
				\end{bmatrix}$ is a linear combination of the rows of $A$. The row space of $A$ is the column space of $A^T$. By the the fundamental theorem of linear algebra $C(A^T) \perp N(A)$, so the vectors $\begin{bmatrix}
				    1 \\
				    1 \\
				    1 \\
				\end{bmatrix} 
				\begin{bmatrix}
				    1 & 1 & 1
				\end{bmatrix}$ must be orthogonal. To test this we use the dot product if the result is $\vec{0}$, then they are orthogonal in this case they are not and this is not possible.
	
		\end{solution}				
	\end{parts}

	\question Picture \(\mathbb{R}^3\). Suppose the wall $W$ and the floor $V$ are two subspaces of 3-dimensional space. The floor and wall are not orthogonal because they share a nonzero vector (along the line where they meet). In fact, no two planes in $\mathbb{R}^3$ can be orthogonal. 
	
	\begin{parts}
	\part Consider the two planes described by the column spaces of the following matrices.
	
	\[A= \begin{bmatrix}
	1 & 2 \\
	1 & 3\\
	1 & 2
	\end{bmatrix} 
	\hspace{1cm} 
	B= \begin{bmatrix}
	5 & 4\\ 
	6 & 3 \\ 
	5 & 1
	\end{bmatrix}\] 
	
	Consider the augmented matrix $\begin{bmatrix}A&B\end{bmatrix}$.
	Find a vector in the nullspace of $\begin{bmatrix}A&B\end{bmatrix}$
	and write down the linear combination of vectors which equals zero.
	Use this linear combination to find a vector $\vec{v}$ 
	that is in both $C(A)$ and $C(B)$.
	
	\begin{solution}
							We first start by creating an augmented matrix $C=[AB]$ $C=
				\begin{bmatrix}
				    1 & 2 & 5 & 4 \\
				    1 & 3 & 6 & 3 \\
				    1 & 2 & 5 & 1 \\
				\end{bmatrix}$ using elimination we can get $\begin{bmatrix}
				    1 & 2 & 5 & 4 \\
				    0 & 1 & 1 & -1 \\
				    0 & 0 & 0 & -3 \\
				\end{bmatrix}$. If we find a vector that is in the null space of $C$. We can use the relationship with the null space vector and the columns of C to find a vector in both $A$ and $C$. 
				\newline
				$\begin{bmatrix}
				    1 & 2 & 5 & 4 \\
				    0 & 1 & 1 & -1 \\
				    0 & 0 & 0 & -3 \\
				\end{bmatrix}\vec{x}=\begin{bmatrix}
				    0 \\ 0 \\ 0 \\ 0 \\
				\end{bmatrix}$
				
				$\vec{x}=\begin{bmatrix}
				    -3 \\ -1 \\ 1 \\ 0 \\
				\end{bmatrix}$ So,
				\newline
				$-3\begin{bmatrix}
				    1 \\ -1 \\ 1 \\ 
				\end{bmatrix}_1+ -1\begin{bmatrix}
				    2 \\ 3 \\ 2 \\ 
				\end{bmatrix}_2+1\begin{bmatrix}
				    5 \\ 6 \\ 5 \\ 
				\end{bmatrix}_3+0\begin{bmatrix}
				    4 \\ 3 \\ 1 \\ 
				\end{bmatrix}_4=\vec{0}$. Since the vectors 1 and 2 are linear combinations of $A$ and the vectors 3 and 4 are linear combinations of $B$.  Therefore 
				\newline
				$-3\begin{bmatrix}
				    1 \\ -1 \\ 1 \\ 
				\end{bmatrix}_1+ -1\begin{bmatrix}
				    2 \\ 3 \\ 2 \\ 
				\end{bmatrix}_2=\begin{bmatrix}
				    3 \\ 3 \\ 3
				\end{bmatrix}+\begin{bmatrix}
				    2 \\ 3 \\ 2
				\end{bmatrix}=\begin{bmatrix}
				    5 \\ 6 \\ 5
				\end{bmatrix}$
				\newline
				So $\begin{bmatrix}
				    5 \\ 6 \\ 5
				\end{bmatrix}$ must also be in the column space of $B$ as well in order to equal the $\vec{0}$
			
	\end{solution}
		
	\part We would like to understand when two subspaces \textbf{must} have a common nonzero vector, just like in the case of two planes in $\mathbb{R}^3$.
	Consider subspaces $V$ and $W$ of $\mathbb{R}^n$. 
	If $V$ has dimension $p$ and $W$ has dimension $q$, 
	describe a process generalizing the previous part that will allow us to detect a nonzero vector in both $V$ and $W$.
	What inequality involving $p$, $q$, and $n$ will \emph{guarantee}
	that $V$ and $W$ have a nonzero vector in common?
	
	\begin{solution}
					Each vector in the null space corresponds to a linear combination of the columns of $A$ and columns of $B$ that yields the $\vec{0}$ because of this there has to be a linear dependence between the columns of $A$ and $B$ which means there is exactly 1 intersection for each free variable.  We can define the matrix $C$ as $C=[VW]$ in order for $V$ and $W$ to guarantee intersection at a non zero vector the dimension of  $N(C) > 0$ if $p+q > n$. Otherwise there is not any free variable and can not intersect at some non zero vector.

	\end{solution}	
	
	\end{parts}
	
	
	\question \S 4.2 \# 16 What linear combination of $(1,2,-1)^\intercal$ and  $(1,0,1)^\intercal$ is closest to $\vec{b} = (2,1,1)^\intercal$? 
	
	\begin{solution}
				Given $P=A(A^TA)^{-1}A^T$ and $A\vec{x}=\vec{b}$ we can solve for $\vec{x_0}$ with $PA\vec{x_0}=P\vec{b}$
		\newline
		$A(A^TA)^{-1}A^TA\vec{x_0}=A(A^TA)^{-1}A^T\vec{b}$
	    \newline
	    $IA^TA\vec{x_0}=A^TA(A^TA)^{-1}A\vec{b}$
	    \newline
	    $A^TA\vec{x_0}=A^T\vec{b}$
	    \newline
	    Given the linear combination (1, 2, -1) adn (1, 0, 1) we can construct the matrix $A=\begin{bmatrix}
	        1 & 1 \\
	        2 & 0 \\
	        -1 & 1 \\
	    \end{bmatrix}$.
	    We can use the matrix A and set it up as the equation above and solve for the $\vec{x_0}$.
	    \newline
	    $\begin{bmatrix}
	        1 & 2 & -1 \\
	        1 & 0 & 1 \\
	    \end{bmatrix}
	    \begin{bmatrix}
	        1 & 1 \\
	        2 & 0 \\
	        -1 & 1 \\
	    \end{bmatrix}\vec{x_0}=\begin{bmatrix}
	        3 \\ 3
	    \end{bmatrix}$.
	    So $\vec{x_0}=\begin{bmatrix}
	        \frac{1}{2} \\ \frac{3}{2}
	    \end{bmatrix}$ which are the coefficients of the closest linear combinations of $A$.

	\end{solution}
	
	\question Let $P$ be a projection matrix. 
	
	\begin{parts}
	\part Show that if $P^2 = P$, then $(I-P)^2 = I-P$. 
		
	\begin{solution}
				With the projected vectors $P\vec{x}$ and $(I-P)y$ knowing $P^2=P$ and $P^T=P$. Given this we can set up the equation,
		\newline
		$(P\vec{x})^T(I-P)\vec{y}=\vec{y}$
		\newline
		$(\vec{x}^TP^T)(\vec{y}-P\vec{y})$
		\newline
		$\vec{x}^TP^T\vec{y}-\vec{x}^TP^TP\vec{y}$
		\newline
		$\vec{x}^TP\vec{y}-\vec{x}^TPP\vec{y}$
		\newline
		$\vec{x}^TP\vec{y}-\vec{x}^TP^2\vec{y}$
		\newline
		In order to be orthogonal as stated then the following equation must be true.
		\newline
		$\vec{x}^TP\vec{y}-\vec{x}^TP\vec{y}=0$
		\newline
		Projects onto the $N(A^T)$.

	\end{solution}
			
	\part It turns out that $I-P$ is also a projection matrix. Show that $I-P$ projects onto a space perpendicular to the subspace onto which $P$ projects vectors. (Hint: Use each of $P$ and $I-P$ to project an arbitrary vector. Are the results orthogonal?)
	If $P$ projects onto $C(A)$, then $I - P$ projects onto which subspace? 
	
	\begin{solution}
					Given $y=mx+b$, we can use the given points plug them in to the equation and then create a matrix from them.
			\newline
			$m(1)+b=1$
			\newline
			$m(2)+b=1$
			\newline
			$m(3)+b=2$
			\newline
			$m(4)+b=2$
			\newline
			So, $\begin{bmatrix}
			    1 \\ 2 \\ 3 \\ 4
			\end{bmatrix} 
			\begin{bmatrix}
			    m
			\end{bmatrix}=
			\begin{bmatrix}
			    1 \\ 1 \\ 2 \\ 2
			\end{bmatrix}$
			\newline
			using $A^TAx=A^T\vec{b}$ we get 
			\newline
			$\begin{bmatrix}
			    1 & 2 & 3 & 4
			\end{bmatrix}
			\begin{bmatrix}
			    1 \\ 2 \\ 3 \\ 4
			\end{bmatrix}
			\begin{bmatrix}
			    m
			\end{bmatrix}=\begin{bmatrix}
			    1 & 2 & 3 & 4
			\end{bmatrix}
			\begin{bmatrix}
			    1 \\ 1 \\ 2 \\2
			\end{bmatrix} \rightarrow 30m=17$.
			\newline
			So $m=\frac{17}{30}$, further $y=\frac{17}{30}x+b$ and since it passes through the origin $b=0$. Finally we get $y=\frac{17}{30}x$
	\end{solution}
	
	\end{parts}
	
	\question 	
	
	\begin{parts}
	\part Find the best line through the origin which fits the points $(1,1),(2,1),(3,2),(4,2)$. 
	
	\begin{solution}
					Given $y=mx+b$, we can use the given points plug them in to the equation and then create a matrix from them.
			\newline
			$m(1)+b=1$
			\newline
			$m(2)+b=1$
			\newline
			$m(3)+b=2$
			\newline
			$m(4)+b=2$
			\newline
			So, $\begin{bmatrix}
			    1 \\ 2 \\ 3 \\ 4
			\end{bmatrix} 
			\begin{bmatrix}
			    m
			\end{bmatrix}=
			\begin{bmatrix}
			    1 \\ 1 \\ 2 \\ 2
			\end{bmatrix}$
			\newline
			using $A^TAx=A^T\vec{b}$ we get 
			\newline
			$\begin{bmatrix}
			    1 & 2 & 3 & 4
			\end{bmatrix}
			\begin{bmatrix}
			    1 \\ 2 \\ 3 \\ 4
			\end{bmatrix}
			\begin{bmatrix}
			    m
			\end{bmatrix}=\begin{bmatrix}
			    1 & 2 & 3 & 4
			\end{bmatrix}
			\begin{bmatrix}
			    1 \\ 1 \\ 2 \\2
			\end{bmatrix} \rightarrow 30m=17$.
			\newline
			So $m=\frac{17}{30}$, further $y=\frac{17}{30}x+b$ and since it passes through the origin $b=0$. Finally we get $y=\frac{17}{30}x$
			

	\end{solution}
	
	\part Find the best parabola which fits the points $(-1, 1/4), (1,1/4), (2,1), (3,2).$
	
	\begin{solution}
					Given $y=ax^2+bx+c$ we can use the given points to fill in the x and y variables then construct a matrix from them.
			\newline
			$a(-1)^2+b(-1)+c=\frac{1}{4}$
			\newline
			$a(1)^2+b(1)+c=\frac{1}{4}$
			\newline
			$a(2)^2+b(2)+c=1$
			\newline
			$a(3)^2+b(3)+c=2$
			\newline,
			we get, $\begin{bmatrix}
			    1 & -1 & 1 \\
			    1 & 1 & 1 \\
			    4 & 2 & 1 \\
			    9 & 3 & 1 \\
			\end{bmatrix}
			\begin{bmatrix}
			    a \\ b \\ c
			\end{bmatrix}=
			\begin{bmatrix}
			    \frac{1}{4} \\
			    \frac{1}{4} \\
			    1 \\
			    2 \\
			\end{bmatrix}$.
			\newline
			Using $A^TA\vec{x}=A^T\vec{b}$ we get, 
			\newline
			$
			\begin{bmatrix}
			    1 & 1 & 4 & 9 \\
			    -1 & 1 & 2 & 3 \\
			    1 & 1 & 1 & 1 \\
			\end{bmatrix}
			\begin{bmatrix}
			    1 & -1 & 1 \\
			    1 & 1 & 1 \\
			    4 & 2 & 1 \\
			    9 & 3 & 1 \\
			\end{bmatrix}
			\begin{bmatrix}
			    a \\ b \\ c
			\end{bmatrix}=
			\begin{bmatrix}
			    1 & 1 & 4 & 9 \\
			    -1 & 1 & 2 & 3 \\
			    1 & 1 & 1 & 1 \\
			\end{bmatrix}
			\begin{bmatrix}
			    \frac{1}{4} \\
			    \frac{1}{4} \\
			    1 \\
			    2 \\
			\end{bmatrix}$
			\newline
			$\begin{bmatrix}
			    99 & 35 & 15 \\
			    35 & 15 & 5 \\
			    15 & 5 & 4 \\
			\end{bmatrix}
			\begin{bmatrix}
			    a \\ b \\ c
			\end{bmatrix}=
			\begin{bmatrix}
			    22.5 \\ 8 \\ 3.5
			\end{bmatrix}
			$
		    So,
		    $99a +35b +15c = 22.5$
		    \newline
		    $35a+15b+5c=8$
		    \newline
		    $15a+5b+4c=3.5$
		    \newline
		    Therefore $\begin{tabular}{c} a=.21 \\ b=.0238 
		    \\ c=.0568\end{tabular}$.
		    Finally we get $y=.21x^2+.0238x+.0568$!

	\end{solution}
	
	\end{parts}
	
	\question Suppose that we would like to predict the incubation period for the SARS-CoV2 virus (the one that has us all doing this assignment remotely).
	Our prediction for the incubation period will be denoted $x$, in days.
	We investigate $m$ COVID-19 cases and record the incubation period in each case by $b_i$.
	This gives us a data vector: $\vec{b} = (b_1, \ldots, b_m)^\intercal$.
	If our prediction for the incubation period was a perfect prediction,
	then each incubation period we recorded would be $x$.
	That is, the following system would hold:
	\[
	\begin{cases}
	x=b_1, \\
	x=b_2, \\
	\vdots \\
	x=b_m
	\end{cases}
	\] 
	In matrix form, this is $A\vec{x}=\vec{b}$, where $\vec{x}$ is a $1 \times 1$ matrix and $A = \begin{bmatrix}1 \\ \vdots \\ 1\end{bmatrix}$.
	
	\begin{parts}
	\part Explain when the system $A\vec{x} = \vec{b}$ has a solution.

	\begin{solution}
						Because $\vec{b}$ has to be in the column space of $A\vec{x}$ must be a scalar of $\vec{b}$. So $\vec{b}$ must be a $m\times1$ matrix.

	\end{solution}
			
	\part Suppose the system from part (a) has no solution (since it probably doesn't). Use least squares to find the best possible solution to the system, call the solution $\hat{x}$. 

	\begin{solution}
						Using the projection equation and multiplying both sides by $A^T$ we can get the following.

				\begin{gather*}
				    A\hat{x}=P\vec{b}\\
				    A\hat{x}=A(A^TA)^{-1}A^T\vec{b} \\
				    A^TA\hat{x}=A^TA(A^TA)^{-1}A^T\vec{b} \\
				    A^TA\hat{x}=A^T\vec{b} \\
				\end{gather*}
				Now if we multiply by $(A^TA)^{-1}$ we can reduce the left side of the equation.
				\begin{gather*}
				    (A^TA)^{-1}A^TA\hat{x}=(A^TA)^{-1}A^T\vec{b} \\
				    \hat{x}=(A^TA)^{-1}A^T\vec{b}
				\end{gather*}


	\end{solution}
	
	\part In statistics, we measure the error in our prediction $\hat{x}$
	by adding the sum of the squared errors.
	This quantity is called the \emph{variance}.
	Each squared error term is $(b_i-\hat{x})^2$,
	and adding these up is the same as computing
	$\lVert\vec{b} - A \hat{x}\rVert^2$.
	Compute the variance using this last formula.
	
	\begin{solution}
					Given $\vec{e}=\vec{b}-A\hat{x}$ we can infer $|\vec{e}|=|\vec{b}-A\hat{x}|$. So $|\vec{b}-A\hat{x}|^2=|\vec{e}|^2$ and $|\vec{e}|^2=\vec{e}^T\vec{e}$. Since $\hat{x}$ is a scaler of $A$ and $A$ is comprised of 1's so $A$ becomes a vector of $\hat{x}$ entries so $$|\vec{e}|^2=\sum_{n=1}^{m}(b_n-\hat{x})^2$$

	\end{solution}
		
	\part Suppose that we record data from three COVID-19 cases
	and find that the incubation periods are 1, 2, and 6 days.
	That is, $\vec{b} = (1,2,6)^\intercal$. 
	Use least squares to find the best solution to $A \vec{x} = \vec{b}$. 
	How is the best solution related to the entries of $\vec{b}$?
			
	\begin{solution}
						Given $\vec{b}=\begin{bmatrix}1 \\ 2 \\ 6\end{bmatrix}$ and using the equation $A^TA\hat{x}=A^T\vec{b}$ we can get, \newline
				    \begin{gather*}
				    \begin{bmatrix}1 & 1 & 1 \end{bmatrix}
				    \begin{bmatrix}1 \\ 1 \\ 1 \end{bmatrix}
				    \hat{x}=
				    \begin{bmatrix}1 & 1 & 1 \end{bmatrix}
				    \begin{bmatrix}1 \\ 2 \\ 6 \end{bmatrix}
				    \\
				    3\hat{x}=9
				    \\
				    \hat{x}=3
				    \end{gather*}
				    So the best solution is $$\left(\sum_{n=1}^3\vec{b}_n \right)\frac{1}{3}$$

	\end{solution}
	
	\part We know that the error vector is orthogonal to $A \hat{x}$. 
	Demonstrate that in the example from the previous part,
	$A \hat{x}$ is orthogonal to $\vec{e}$.
	
	\begin{solution}
						This can be shown by using the equation $\vec{e}=\vec{b}-A\vec{x}$ we get the following equation that we can solve for. \newline
				\begin{gather*}
				    \begin{bmatrix}
				    -2 \\ -1 \\ 3
				    \end{bmatrix}_{\vec{e}}
				    =
				     \begin{bmatrix}
				    1 \\ 2 \\ 6
				    \end{bmatrix}_{\vec{b}}
				    -
				     \begin{bmatrix}
				    3 \\ 3 \\ 3
				    \end{bmatrix}_{A\vec{x}}
				  \\
				    \begin{bmatrix}
				    3 & 3 & 3
				    \end{bmatrix}
				    \begin{bmatrix}
				    -2 \\ -1 \\ 3    
				    \end{bmatrix}
				   \\
				    -6-3+9=0
				\end{gather*}  
	\end{solution}
				
	\end{parts}

	\question
	
	\begin{parts}
	\part Find orthonormal vectors $q_1, q_2, q_3$ such that $q_1, q_2$ span the column space of 
		\[A = \begin{bmatrix}
		4 & 1 \\ -4 & 5 \\ -2 & 1
		\end{bmatrix}\]
		
	\begin{solution}
					To find $q_1$ we use $q_1=$ $v_1\over{|v_1|}$ to find $q_2$ we use $q_2=v_2-q_1^T v_2q_1$
			\begin{gather*}
			    q_1=\begin{bmatrix}4 \\ -4 \\ -2 \end{bmatrix}1/6 = \begin{bmatrix}2/3 \\ -2/3 \\ -1/3 \end{bmatrix} 
			    \\
			    v_2=\begin{bmatrix}1 \\ 5 \\ 1 \end{bmatrix}-\begin{bmatrix}2/3 & -2/3 & -1/3 \end{bmatrix} \begin{bmatrix}1 \\ 5 \\ 1 \end{bmatrix}\begin{bmatrix}2/3 \\ -2/3 \\ -1/3 \end{bmatrix}=\begin{bmatrix}3 \\ 3 \\ 0\end{bmatrix}
			    \\
			    q_2=v_2/|v_2|=\begin{bmatrix}1/\sqrt{2} \\ 1/\sqrt{2} \\ 0\end{bmatrix}
			\end{gather*}
			Per the fundamental theorem of linear algebra we know $N(A^T)\perp C(A)$. Using this we can solve for $N(A^T)$ using $A^T=\begin{bmatrix} 4 & -4 & -2 \\ 1 & 5 & 1\end{bmatrix}$ we get 
			$rref(A^T)=R=
			\begin{bmatrix} 
			1 & 0 & -1/4 \\ 
			0 & 1 & 1/4
			\end{bmatrix}$ 
			Using this matrix we can solve for the 
			$N(A^T)=
			\begin{bmatrix} 
			1/4 \\ -1/4 \\ 1 
			\end{bmatrix}$ 
			which when normalized is the vector. $q_3$ \newline
			$q_3 = \begin{bmatrix}\sqrt{2}/6 \\ -\sqrt{2}/6 \\ 2\sqrt{2}/3 \end{bmatrix}$

	\end{solution}	
	
	\part Which of the four fundamental subspaces contains $q_3$? 

	\begin{solution}
							Using A to find the left null-space we get a vector which when normalized gives us the vector $q_3$.

	\end{solution}	

	\part Solve $A \vec{x} = (1,2,7)$ by least squares.

	\begin{solution}
				    Using the equation $A^TA\vec{x}=A^T\vec{b}$ we get the following. 
			    \begin{gather*}
			        \begin{bmatrix}
			        4 & -4 & -2 \\
			        1 & 5 & 1 \\
			        \end{bmatrix}
			        \begin{bmatrix}
			        4 & 1 \\
			        -4 & 5 \\
			        -2 & 1 \\
			        \end{bmatrix}
			        \vec{x}
			        =
			        \begin{bmatrix}
			        4 & -4 & -2 \\
			        1 & 5 & 1 \\
			        \end{bmatrix}
			        \begin{bmatrix}
			        1 \\
			        2 \\
			        7 \\
			        \end{bmatrix} 
			        \\
			        \begin{bmatrix}
			        36 & -18 \\
			        -18 & 27 \\
			        \end{bmatrix}
			        \vec{x}= \begin{bmatrix}
			        -18 \\
			        9 \\
			        \end{bmatrix}
			        \\
			        \begin{bmatrix}
			        36 & -18 \\
			        0 & -18 \\
			        \end{bmatrix}
			        \vec{x}= \begin{bmatrix}
			        -18 \\
			        9 \\
			        \end{bmatrix} 
			        \\
			        36x_1-18x_2=-18
			        \\
			        18x_2=9
			        \\
			        x_2=1/2
			        \\
			        x_1=-1/4
			    \end{gather*}
			    So $\vec{x}=(-1/4, 1/2)$

	\end{solution}	
	\end{parts}
	
\end{questions}





\end{document}