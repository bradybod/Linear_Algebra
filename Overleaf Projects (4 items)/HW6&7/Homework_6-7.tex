\documentclass[]{exam}

\usepackage{amsmath, amssymb}
\usepackage[margin=1in]{geometry}
\usepackage[en-US, showdow]{datetime2}
\usepackage{graphicx}
\usepackage{multicol,enumitem}

\newenvironment{smallbmatrix}
{\left[\begin{smallmatrix}}
{\end{smallmatrix}\right]}


\title{Homework 6 \& 7 \\
	Due \DTMdate{2020-04-24} %Add a % at the beginning of the line to remove the due date from the title (or just delete this line entirely) 
	}
\date{ %\today %You can remove the first % in this line to show the current days date.
	}
\author{Name: Brady Bodily \\
		\footnotesize Collaborators: Mitch Pound, Danny Clyde, Jaxon Willard, Jared Horton  \\
		\footnotesize Recitation Section: 504
		%You should probably change this to your actual name, at least if you want credit.
		}
\begin{document}
\maketitle

\printanswers %comment out this line to hide your answers.

\begin{questions}

%HOMEWORK 6
\question \S 5.1\# 1-6
Complete the following problems. Give a \textit{one-sentence} justification of each (in other words, don't spend an eternity answering these questions).

\begin{parts}
	\part If a 4 by 4 matrix has determinant $\det A = \frac{1}{2}$,
	what are $\det(3A)$, $\det(-A)$, and $\det(A^3)$?
	
	\begin{solution}
		\input{p1a}
	\end{solution}
	
	\part If a 3 by 3 matrix has determinant $\det A = -1$,
	what are $\det(\frac{1}{2}A)$, $\det(-A)$, and $\det(A^2)$?
	
	\begin{solution}
        \\
$det(\frac{1}{2}A) = \frac{1}{2}^3det(A) = -\frac{1}{8}$
\\
$det(-A) = -1^3det(A) = 1$
\\
$det(A^2) = det(A)^2 = 1$
    \end{solution}
	
	\part State whether each is true or false, with a one-sentence reason if true or a counterexample if false.
	\begin{subparts}
		\subpart The determinant of $I+A$ is $1+\det A$.
		\subpart The determinant of $ABC$ is $|A||B||C|$.
		\subpart The determinant of $4A$ is $4|A|$.
	\end{subparts}
	
	\begin{solution}
        \\
i. $\begin{bmatrix}1 & 0 \\ 0 & 1 \end{bmatrix}+\begin{bmatrix}3 & 2 \\ 1 & 4\end{bmatrix} =  \begin{bmatrix}4 & 2 \\1 & 5\end{bmatrix}$
\\

    \end{solution}
	
	\part Let $J_n$ denote the $n \times n$ ``reverse identity matrix''.
	$J_3$ and $J_4$ are given as examples below:
	\[ J_3 = \begin{bmatrix} 0&0&1 \\ 0&1&0 \\ 1&0&0 \end{bmatrix}, \qquad
       J_4 = \begin{bmatrix} 0&0&0&1 \\ 0&0&1&0 \\ 0&1&0&0 \\ 1&0&0&0 \end{bmatrix}. \]
    \begin{subparts}
    	\subpart Which row or column exchanges show that $|J_3|=-1$ and $|J_4|=1$?
    	\subpart Compute $|J_4|$, $|J_5|$, and $|J_6|$.
    	\subpart Propose a rule for $|J_n|$ and use it to compute $|J_{101}|$.
    \end{subparts}
    
    \begin{solution}
		\input{p1d}
	\end{solution}
    
    \item Use the fact that the determinant is linear in each row 
    to show that $\det A = 0$ whenever $A$ has a row of zeros.
    
    \begin{solution}
		\input{p1e}
	\end{solution}
\end{parts}

\question Recall that an orthogonal matrix $Q$ is a matrix satisfying $Q^\intercal Q = I$. If $Q$ is a square orthogonal matrix, prove that $\det Q = \pm 1$.

(\textbf{Hint:} Use the fact that $|A^\intercal| = |A|$ and $|AB|=|A||B|$.)

\begin{solution}
	You can write your solution here!
\end{solution}

	
\question \S 5.1 \# 28 State if the following statements are true are false. If true provide an explanation, if false give a 2 by 2 counterexample.
	
\begin{parts}
	\part If $A$ is not invertible, then $AB$ is not invertible. 
	
	\begin{solution}
		You can write your solution here!
	\end{solution}
	
	\part The determinant of $A$ is always the product of its pivots.
	
	\begin{solution}
		You can write your solution here!
	\end{solution}
	
	\part The determinant of $A-B$ equals $|A| - |B|$.
	
	\begin{solution}
		You can write your solution here!
	\end{solution}
	
	\part $AB$ and $BA$ have the same determinant. 
	
	\begin{solution}
		You can write your solution here!
	\end{solution}
\end{parts}

\question Consider a 3 by 3 matrix whose entries are all 1 or -1. Show the maximum determinant of such a matrix is 4. Give an example with determinant equal to 4. 

(\textbf{Hint:} First explain why the determinant is less than or equal to 6. Then individually rule out why the determinant cannot be 6 and why the determinant cannot be 5.)
	
\begin{solution}
	You can write your solution here!
\end{solution}

\question Suppose that $A \vec{x} = \lambda\vec{x}$. Please prove the following. 
	
\begin{parts}
	\part 	$\lambda^2$ is an eigenvalue of $A^2$.

	\begin{solution}
		\\
		\newline
		Given $A \vec{x} = \lambda\vec{x}$ we can prove this with basic algebra.
		\begin{center}
		    $A^2 \vec{x}= \lambda^2\vec{x}$
		    \newline
		    $A^2 \vec{x}= AA\vec{x}$
		    \newline
		    $AA \vec{x} = A\lambda\vec{x}$
		    \newline
		    $A \lambda \vec{x} = \lambda A\vec{x}$
		    \newline
		    $\lambda A \vec{x} = \lambda\lambda\vec{x}$
		    \newline
		    $\lambda \lambda \vec{x} = \lambda^2\vec{x}$
		    \newline
		\end{center}
		Therefore proving $\lambda^2$ is an eigenvalue of $A^2$.

	\end{solution}
	
	\part  $\lambda^{-1}$ is an eigenvalue of $A^{-1}$. (This means the eigenvalues of inverses are reciprocals!)
	
	\begin{solution}
		\newline
		\begin{center}
		    $A\vec{x}=\lambda\vec{x}$
		    \newline
		    $A^{-1}A\vec{x}=A^{-1}\lambda\vec{x}$
		    \newline
		    $I\vec{x}=A^{-1}\lambda\vec{x}$
		    \newline
		    $\vec{x}=A^{-1}\lambda\vec{x}$
		    \newline
		    $\lambda^{-1}\vec{x}=\lambda^{-1}A^{-1}\lambda\vec{x}$
		    \newline
		    $\lambda^{-1}\vec{x}=A^{-1}\vec{x}$
		    \newline
		\end{center}
		Therefore proving $\lambda^{-1}$ is an eigenvalue of $A^{-1}$.

	\end{solution}
	
	\part $\lambda + k$ is an eigenvalue of $A + kI$. (So changing all the diagonals by $k$ shifts the eigenvalues by $k$.)
	
	\begin{solution}
			For $(A^T = -A)$ to be true then the diagonal has to be 0's. So,
				\newline
				$
				\begin{bmatrix}
				0 & a & b \\
				-a & 0 & c \\
				-b & -c & 0
				\end{bmatrix}
				=
				\begin{bmatrix}
				0 & a & 0 \\
				-a & 0 & 0 \\
				0 & 0 & 0
				\end{bmatrix}
				+
				\begin{bmatrix}
				0 & 0 & b \\
				0 & 0 & 0 \\
				-b & 0 & 0
				\end{bmatrix}
				+
				\begin{bmatrix}
				0 & 0 & 0 \\
				0 & 0 & c \\
				0 & -c & 0
				\end{bmatrix}
				$
				Since they are linearly independent then span is, 
				\newline
				$
				Span \left(
				\begin{bmatrix}
				0 & a & 0 \\
				-a & 0 & 0 \\
				0 & 0 & 0
				\end{bmatrix}
				,
				\begin{bmatrix}
				0 & 0 & b \\
				0 & 0 & 0 \\
				-b & 0 & 0
				\end{bmatrix}
				,
				\begin{bmatrix}
				0 & 0 & 0 \\
				0 & 0 & c \\
				0 & -c & 0
				\end{bmatrix}
				\right)
				$
				Since there are 3 matrices in the span the dimension is 3, and $(A^T = -A)$ is true for each of the matrices in the span. Any linear combination of these matrices would result in a symmetric matrix.
	\end{solution}
\end{parts}

\question Recall that the characteristic polynomial $\det(A-\lambda I)$ can always be written $(\lambda_1 - \lambda)(\lambda_2 - \lambda) \cdots (\lambda_n - \lambda)$ where each $\lambda_i$ is a root of $\det(A-\lambda I)$. Use this to prove that det$(A) = \lambda_1\lambda_2\cdots\lambda_n$.

(\textbf{Hint:} Make a good choice to substitute for $\lambda$.)

\begin{solution}
		Given, $C(A)=Span \left( 
	\begin{bmatrix}
	1 \\ 2 \\ 4
	\end{bmatrix}
	,
	\begin{bmatrix}
	2 \\ 2 \\ 1
	\end{bmatrix}
	\right)$
	and 
	$C(A^T)=Span \left( 
	\begin{bmatrix}
	1 \\ 0
	\end{bmatrix}
	,
	\begin{bmatrix}
	1 \\ 1
	\end{bmatrix}
	\right)$
	We can construct the outer product from the column space of $C(A)$ and $C(A^T)$. So,
	\newline
	$
	\begin{bmatrix}
	1 \\ 2 \\ 4
	\end{bmatrix}
	\begin{bmatrix}
	1 & 0
	\end{bmatrix}
	+
	\begin{bmatrix}
	2 \\ 2 \\ 1
	\end{bmatrix}
	\begin{bmatrix}
	1 & 1
	\end{bmatrix}
	=A=
	\begin{bmatrix}
	1 & 0 \\
	2 & 0 \\
	4 & 0
	\end{bmatrix}
	+
	\begin{bmatrix}
	2 & 2 \\
	2 & 2 \\
	1 & 1
	\end{bmatrix}
	=
	\begin{bmatrix}
	3 & 2 \\
	4 & 2 \\
	5 & 1
	\end{bmatrix}
	$
	\newline
	Using the outer product vectors we can create 2 matrices and verify that the product equals $A$.
	$
	\begin{bmatrix}
	1 & 2 \\
	2 & 2 \\
	4 & 1
	\end{bmatrix}
	\begin{bmatrix}
	1 & 0 \\
	1 & 1 \\
	\end{bmatrix}
	=
	\begin{bmatrix}
	3 & 2 \\
	4 & 2 \\
	5 & 1
	\end{bmatrix}
	=A
	$
\end{solution}

\question Diagonalize the matrix $A$ below. Then use the diagonalization to show that the given formula for $A^k$ holds.
	
\begin{align*}
	A = \begin{bmatrix}
	2 & -1 \\ -1 & 2
	\end{bmatrix} & &
	A^k = \frac{1}{2}\begin{bmatrix}
	1 + 3^k & 1 - 3^k \\ 1 - 3^k & 1 + 3^k
	\end{bmatrix}
\end{align*}

\begin{solution}
	You can write your solution here!
\end{solution}
	
\question One important theorem in linear algebra is the \emph{Cayley-Hamilton Theorem}. This theorem states that if $p(\lambda) = \det(A-\lambda I)$ is the characteristic polynomial for $A$, then $p(A)=0$. This is surprising because $\lambda$, which is a scalar, is replaced by the matrix $A$ in the polynomial and it evaluates to 0. 
	
\begin{parts}
	\part Verify that the Cayley-Hamilton Theorem works for \[A= \begin{bmatrix}
	a & b \\ 0 & d
	\end{bmatrix}.\]
	That is, find the characteristic polynomial for $A$,
	then replace $\lambda$ with $A$
	and verify that the resulting expression simplifies to the zero matrix.
	
	\begin{solution}
		\newline
		First start by calculating the matrix $A-\lambda I$.  So we get,
		\begin{center}
		    $
		    \begin{bmatrix}
		    a & b \\ 0 & d
		    \end{bmatrix}-
		    \lambda
		    \begin{bmatrix}
		    1 & 0 \\ 0 & 1
		    \end{bmatrix}=
		    \begin{bmatrix}
		    a-\lambda & b \\ 0 & d-\lambda
		    \end{bmatrix}
		   $
		\end{center}.
	    We can now take the determinate of the result.
	    \newline
	    \begin{center}
	        $\begin{vmatrix}
	        a-\lambda & b \\ 0 & d-\lambda
	        \end{vmatrix}=
	        (a-\lambda)(d-\lambda)(-b0)$
	        \newline
	        $(a-\lambda)(d-\lambda)=\lambda^2-a\lambda-d\lambda+ad$
	        \newline
	    \end{center}
	    We can now plug A in for $\lambda_1$.
	    \begin{center}
	    $
	        \begin{bmatrix}
	            a & b \\ 0 & d
	        \end{bmatrix}^2-
	        a\begin{bmatrix}
	            a & b \\ 0 & d
	        \end{bmatrix}-
	        d\begin{bmatrix}
	            a & b \\ 0 & d
	        \end{bmatrix}
	        + ad
	    $
	    \newline
	    $
	        \begin{bmatrix}
	            a & b \\ 0 & d
	        \end{bmatrix}
	        \begin{bmatrix}
	            a & b \\ 0 & d
	        \end{bmatrix}-
	        \begin{bmatrix}
	            a^2 & ab \\ 0 & ad(I)
	        \end{bmatrix}-
	        \begin{bmatrix}
	            ad & bd \\ 0 & d^2
	        \end{bmatrix}
	        + ad
	    $
	    \newline
	    $
	        \begin{bmatrix}
	            a^2 & ab+bd \\ 0 & d^2
	        \end{bmatrix}-
	        \begin{bmatrix}
	            a^2 & ab \\ 0 & ad(I)
	        \end{bmatrix}-
	        \begin{bmatrix}
	            ad & bd \\ 0 & d^2
	        \end{bmatrix}
	        + ad
	    $
	    \newline
	    $
	        \begin{bmatrix}
	            0 & ab+bd \\ 0 & d^2
	        \end{bmatrix}-
	        \begin{bmatrix}
	            0 & ab \\ 0 & ad(I)
	        \end{bmatrix}-
	        \begin{bmatrix}
	            ad & bd \\ 0 & d^2
	        \end{bmatrix}
	        + ad
	    $
	    \newline
	    $
	        \begin{bmatrix}
	            0 & ab+bd \\ 0 & d^2
	        \end{bmatrix}-
	        \begin{bmatrix}
	            0 & ab \\ 0 & ad(I)
	        \end{bmatrix}-
	        \begin{bmatrix}
	            ad & bd \\ 0 & d^2
	        \end{bmatrix}
	        + ad
	    $
	    \newline
	    $
	        \begin{bmatrix}
	            0 & bd \\ 0 & d^2
	        \end{bmatrix}-
	        \begin{bmatrix}
	            0 & 0 \\ 0 & ad(I)
	        \end{bmatrix}-
	        \begin{bmatrix}
	            ad & bd \\ 0 & d^2
	        \end{bmatrix}
	        + ad
	    $
	    \newline
	    $
	        \begin{bmatrix}
	            0 & 0 \\ 0 & 0
	        \end{bmatrix}-
	        \begin{bmatrix}
	            0 & 0 \\ 0 & ad(I)
	        \end{bmatrix}-
	        \begin{bmatrix}
	            ad & 0 \\ 0 & 0
	        \end{bmatrix}
	        + ad
	    $
	    \newline
	    $
	     \begin{bmatrix}
	            -ad & 0 \\ 0 & -ad
	        \end{bmatrix}+ ad
	        \begin{bmatrix}
	        1&0\\0&1
	        \end{bmatrix}
	    $
	    \newline
	    $
	    \begin{bmatrix}
	            -ad & 0 \\ 0 & -ad
	        \end{bmatrix}+
	        \begin{bmatrix}
	        ad&0\\0&ad
	        \end{bmatrix}=0
	    $
	    \newline
	    \end{center}

	\end{solution}
	
	\part Together, we will prove the Cayley-Hamilton Theorem in the special case where $A$ is a diagonalizable $n\times n$ matrix.
	Assume, for the rest of this problem, that $A$ has been diagonalized as
	$A=S\Lambda S^{-1}$.
	
	\begin{subparts}
		\subpart Let $\lambda_1,\lambda_2,\dots,\lambda_n$
		be the eigenvalues of $A$.
		Use the expression $A=S\Lambda S^{-1}$
		to argue that $\lambda_i I - A = S(\lambda_i I - \Lambda) S^{-1}$.
		
		\begin{solution}
			\\
Given $A=S\Lambda S^{-1}$, we will start with $S(\lambda_i I-\Lambda)S^{-1}$.  \\
From here we will distribute the $S$ and $S^{-1}$.
\\
		\begin{center}
		    $(\lambda_i SI-S\Lambda)S^{-1}$
		    \newline
		    $(\lambda_i SIS^{-1}-S\Lambda S^{-1})=\lambda_i I-A$
		    \newline
		    $(\lambda_i I-A)(\lambda_2I-A)...(\lambda_n I-A)=0$
		    \newline
		\end{center}
		Therefore proving the statement.

		\end{solution}
		
		\subpart Recall that the characteristic polynomial is
		$p(\lambda) = \det(A-\lambda I)$.
		Write a brief explanation about why
		$\lambda_1,\dots,\lambda_n$ are the roots of this polynomial.
		Then explain why this means we can factor
		$p(\lambda) = (\lambda_1-\lambda)(\lambda_2-\lambda)\cdots(\lambda_n-\lambda)$.
		
		\begin{solution}
			\input{P8C}
		\end{solution}
		
		\subpart Let us substitute $A$ into the (factored) characteristic polynomial, as follows:
		$p(A) = (\lambda_1 I-A)(\lambda_2 I-A)\cdots(\lambda_n I-A)$.
		Use your answer from the first part to simplify this expression 
		and explain why we get zero.
		
		\begin{solution}
			You can write your solution here!
		\end{solution}
	\end{subparts}
\end{parts}

%HOMEWORK 7

\question Let $S$ be a symmetric matrix ($S=S^T$).
Using the argument in parts (a)-(d),
we will prove that one can always choose the eigenvectors of $S$ to be orthonormal.

\begin{parts}
	\part Let's first start with a simple case. 
	Let $\lambda_1$ be a nonzero eigenvalue of $S$ with eigenvector $\vec{x}$
	and assume that $0$ is an eigenvalue of $S$ with eigenvector $\vec{y}$.
	That is, $S\vec{x} = \lambda_1\vec{x}$ and $S\vec{y} = 0\vec{y}$ and $\lambda_1 \neq 0$. 
	We will ignore the other eigenvalues and eigenvectors of $S$ for now.
	Argue that $\vec{x}$ and $\vec{y}$ are orthogonal in this case.
	
	(\textbf{Hint:} You need to compute $\vec{y}^\intercal \vec{x}$.
	Manipulate this expression by multiplying by a fancy form of one involving $\lambda_1$.)
	
	\begin{solution}
		Given $\lambda_1 \neq 0$, $S=S^T$, $C(S)=C(S^T), N(S)\perp C(S^T)$,
	    \begin{center}
	        $\vec{x}\epsilon(C(S)$
	        \newline
	        $\vec{y}\epsilon(S)$
	        \newline
	        $\lambda_2=0$        
	        \newline
	        $S\vec{x}=\lambda_1\vec{x}$ 
	        \newline
	        $S\vec{y}=0y$          
	        \newline
	        $\vec{x}\perp\vec{y}$
	        \newline
	        
	    \end{center}

	\end{solution}
	
	\part Now suppose that $S\vec{x} = \lambda_1\vec{x}$ and $S\vec{y} = \lambda_2\vec{y}$, where $\lambda_1 \ne \lambda_2$. 
	Consider a shifted matrix $S' = S - \lambda_2 I$. 
	Argue $\lambda_1-\lambda_2$ and $0$ are among the eigenvalues of $S'$
	with corresponding eigenvectors $\vec{x}$ and $\vec{y}$.
	Use part (a) to argue that $\vec{x}$ and $\vec{y}$ are orthogonal.
	
	\begin{solution}
		 Using 1c it shows that if you shift a matrix by a value you also shift the eigenvalues without changing the eigenvectors.
	    \begin{center}
	        $(S-\lambda_2I)\vec{x}=(\lambda_1-\lambda_2)\vec{x}$
	        \newline
	        $(S-\lambda_2I)\vec{y}=(\lambda_1-\lambda_2)\vec{y}$
	        \newline
	    \end{center}
	    Using 5a we see that the Right hand side of the equation goes to 0, so $\vec{x}\perp\vec{y}$
	   
	\end{solution}
	
	\part \textbf{Summary so far:} We have shown that if $\vec{x}$ and $\vec{y}$ are eigenvectors of $S$ corresponding to \textit{distinct} eigenvalues of $S$, then $\vec{x}\perp\vec{y}$. We can then normalize each vector to make them orthonormal. 
	
	(You don't need to do anything for this part, but make sure you understand what's happening.)
	
	\part We've handled the case where eigenvalues are distinct. 
	All that's left to do is show that when we have a repeated eigenvalue, the eigenvectors for that eigenvalue can be chosen orthonormal.
	Suppose $S$ has a repeated eigenvalue $\lambda$ with eigenvectors $\vec{x}_1,\vec{x}_2,\dots,\vec{x}_n$.
	What procedure ensures that we can use these to find an orthonormal set of eigenvectors with eigenvalue $\lambda$?
	
	\begin{solution}
		Assuming S is diagonalizable, 5a, and 5b, prove that eigenvectors are orthonormal. Using grahmschmidt then we can force the eigenvalues to be orthonormal. $(A-\lambda I)\vec{x}=0$ we can use grahmschmidt to find orthonormal vectors.

	\end{solution}
\end{parts} 

\question In the previous problem you proved part of the spectral theorem, which states: Every symmetric matrix can be diagonalized as $S = Q \Lambda Q^T$ where $\Lambda$ is a diagonal matrix of real eigenvalues and the columns of $Q$ are orthonormal. (Notice, this is a lot more special than just a regular diagonalization, since orthogonal matrices are blue-ribbon matrices). 

To see an example of the Spectral Theorem,
diagonalize
\[S = \begin{bmatrix}
1 & 2 \\ 2 & 4
\end{bmatrix}\]
using an orthogonal matrix $Q$.

\begin{solution}
	You can write your solution here!
\end{solution}

\question Let $S$ be a symmetric matrix with positive eigenvalues and $\vec{x}$ be \textit{any} nonzero vector. Prove that $S$ is positive definite. 
That is, $\vec{x}^T S \vec{x} >0$.

(\textbf{Hint:} Use the Spectral Theorem to write $S=Q\Lambda Q^\intercal$. 
Name the vector $\vec{y}=Q^\intercal\vec{x}$ and suppose it is given by $\vec{y}=(y_1,y_2,\dots,y_n)$. Rewrite the expression $\vec{x}^\intercal S \vec{x}$ by substituting appropriately and expand the expression to see that the result is always positive.)

\begin{solution}
	
\end{solution}

\question Let $S$ be a symmetric matrix and $\lambda_{min}$ be the smallest eigenvalue of $S$. In parts (a)-(c), we will show that all the diagonal entries of $S$ are at least as large as $\lambda_{min}$. 
	
Here is an example to help you understand what we are trying to show.
The matrix 
\[\begin{bmatrix}
3  & 2 &  4\\
2  & 0 &  2\\
4  & 2 &  4
\end{bmatrix}\] 
has eigenvalues $-1,-1,8$. The smallest eigenvalue is $\lambda_{min} = -1$, and the diagonal entries are $3$, $0$, and $4$, all of which are bigger than $-1$.

\begin{parts}
	\part  Show that if any matrix $A$ is positive definite, then all the diagonal entries of $A$ are strictly positive.
	
	\begin{solution}
		You can write your solution here!
	\end{solution}

	\part This will be a proof by contradiction.
	So we will suppose that there is a diagonal entry of $S$, 
	say $s_{ii}$, such that $s_{ii} < \lambda_{min}$.
	Prove that $S- s_{ii}I$ has positive eigenvalues. 

	\begin{solution}
		You can write your solution here!
	\end{solution}
	
	\part Use parts (a) and (b) to argue that there is no diagonal entry of $S$ smaller than $\lambda_{min}$.
	
	(\textbf{Hint:} Is the matrix $S- s_{ii}I$ positive definite? What are its diagonal entries?)
	
	\begin{solution}
		You can write your solution here!
	\end{solution}

	\end{parts}

\question Let $A = \begin{bmatrix}
1 & 2 \\ 3 & 6
\end{bmatrix}$. Use a diagonalization of \(A^\intercal A\) to find a singular value decomposition of $A$.

\begin{solution}
	You can write your solution here!
\end{solution}

\question Compute the pseudoinverse of $A = \begin{bmatrix}
2 & 2\\ 1 &1
\end{bmatrix}$.

\begin{solution}
	You can write your solution here!
\end{solution}

\question Consider the linear transformation $\frac{d}{dx}\colon\mathcal{P}_2\to\mathcal{P}_1$ which takes the derivative. 
That is, $\frac{d}{dx}(a+bx+cx^2) = b+2cx$.
We have seen that with respect to the bases $\mathcal{B}_1= \{1,x,x^2\}$ in the domain and $\mathcal{B}_2=\{1,x\}$ in the codomain, the matrix representation of $\frac{d}{dx}$ is
\[ A = \begin{bmatrix}0&1&0\\0&0&2\end{bmatrix}. \]

\begin{parts}
	\part Briefly explain (without referencing the matrix representation $A$) why the linear transformation $\frac{d}{dx}$ is not invertible, then briefly explain why the matrix $A$ is not invertible.
	
	\begin{solution}
		You can write your solution here!
	\end{solution}
	
	\part Compute the pseudoinverse of $A$.
	
	\begin{solution}
		You can write your solution here!
	\end{solution}
	
	\part Consider the linear transformation $\int_0^x \colon \mathcal{P}_1 \to \mathcal{P}_2$ which takes integrals. 
	That is, $\int_0^x (a+bt)\,dt = 0+ax+\frac{1}{2}bx^2$.
	Find the matrix representation with respect to $\mathcal{B}_2$ and $\mathcal{B}_1$.
	
	\begin{solution}
		You can write your solution here!
	\end{solution}
\end{parts}

	
\end{questions}





\end{document}