\documentclass[]{exam}
\usepackage{ dsfont }
\usepackage{amsmath, amssymb}
\usepackage[margin=1in]{geometry}
\usepackage[en-US, showdow]{datetime2}
\usepackage{graphicx}
\usepackage{tikz}
\usepackage{mathabx}
\usepackage{multicol,enumitem}
\newcommand*\circled[1]{\tikz[baseline=(char.base)]{
            \node[shape=circle,draw,inner sep=2pt] (char) {#1};}}

\newenvironment{smallbmatrix}
{\left[\begin{smallmatrix}}
{\end{smallmatrix}\right]}


\title{Homework 3\\
	Due \DTMdate{2020-03-13} %Add a % at the begining of the line to remove the due date from the title (or just delete this line entirely) 
	}
\date{ %\today %You can remove the first % in this line to show the current days date.
	}
\author{Name: Brady Bodily  \\
		\footnotesize Collaborators: Mitch Pound, Danny Clyde, Jaxon Willard  \\
		\footnotesize Recitation Section: \circled{504} /505/506
		%You should probably change this to your actual name, at least if you want credit.
		}
\begin{document}
\maketitle

\printanswers %comment out this line to hide your answers.

\begin{questions}
	\question Suppose $A$ is a $6 \times 4$ matrix. What are all possible dimensions of each of the four fundamental subspaces associated with $A$?
	What are the possible values for the rank of $A$? 
	For each possible rank, explain how many solutions an equation $A\vec{x}=\vec{b}$ might have.
	
	\begin{solution}
	\begin{center}
\begin{tabular}{ c | c c c c c}
\hline
 Rank & 0& 1 & 2 & 3 & 4 \\ 
 \hline
 $dim(C(A))$ & 0 & 1 & 2  & 3 & 4\\  
 $dim(N(A))$ & 4 & 3 & 2 & 1 & 0 \\
 $dim(C(A^T))$ & 0 & 1 & 2 & 3 & 4 \\
 $dim(N(A^T))$ & 6 & 5 & 4 & 3 & 2\\ 
 
\end{tabular}
\end{center}
	\end{solution}
	
	\question We are years into the future and every single one of you has decided to pursue mathematics and is teaching their own Linear Algebra class. Congratulations! You want to make a homework problem that tests the following objectives:
	\begin{itemize}
		\item Students will complete Gaussian elimination with explanation.
		\item Students will identify the pivots of a matrix.
		\item Students will find bases for each of the four fundamental subspaces.
		\item Students will determine whether a matrix is invertible.
		\item Students will find the complete solution to a system $A\vec{x}=\vec{b}$.
	\end{itemize}
	In addition to testing these objectives, you'd like the computations to be fairly friendly to the students (results should mostly consist of integers). Please write a homework question for your students. (Note: For objective 3 above, the four fundamental subspaces should all be somewhat interesting; we don't want the nullspace to only have the zero vector in it, for example). Include a rubric for your homework problem. Feel free to be creative!
	
	\begin{solution}
	Using the row method of matrix multiplication we can easily see which matrices rows are combinations of $C$.  Given the matrices $AB=C \Rightarrow$ {$\begin{bmatrix} \begin{tabular}{cc|cc:c}
		     a & b & e & f & a(e,f)+b(e,f)  \\
		     c & d & g & h & c(g,h)+d(g,h) \\
		\end{tabular} \end{bmatrix}$}, therefore rows of $C$ are a linear combination of the matrix $B$'s rows with coefficients from $A$ resulting in  \begin{center} $R(C) \le R(B)$.\end{center}  Given this we now know that for $B^TA^T= C^T$ the rows of $C^T$ are linear combinations of  rows of $A^T$ and as before $R(C^T)\le R(A^T)$. Since $R(A^T)=R(A)$ we can state $R(C)=min\{R(A),R(B)\}$ Given pivots remain in the same columns and rows during a transformation we know $R(C)=R(C^T)$.  Finally given $R(A)$, $R(B)$, $A$ is $m\times n$, $B$ is $n\times m$, and the matrices are are inversible. We now know \begin{center} $R(C)=min\{R(A),R(B)\}$, \end{center} it is also interesting to note that $R(A)+R(B)-n=min\{R(A),R(B)\}$ so we can also state $R(C)$ is calculated by $R(A)+R(B)-n=R(C)$.

	\end{solution}
	
	\question \S 3.3 \# 34. Suppose that $A$ is a $3 \times 4$ matrix and $\{(2,3,1,0)^\intercal\}$ is a basis for $N(A)$. Please answer the following.
	\begin{enumerate}[label=(\alph*)]
		\item What is the rank of $A$ and the complete solution to $A\vec{x}=\vec{0}$?
		\item What is the row reduced echelon form $R$ of $A$?
		\item Explain why $A\vec{x}=\vec{b}$ can be solved for all $\vec{b}$.
	\end{enumerate}
	
	\begin{solution}
	
\begin{enumerate}[label=(\alph*)]
    \item .\\
    \begin{center}
        $A\Vec{x} = \Vec{0}$ \\
        $A \Vec{x_n}+A \Vec{x_p} = \Vec{b} + \Vec{0}$ \\
        $A(\Vec{x_n} + \Vec{x_p})=\Vec{b}$ \\
        $A\left( \alpha 
        \begin{bmatrix} 
            2 \\
            3 \\
            1 \\
            0 \\
        \end{bmatrix} \right) = \Vec{0}$
        
        $N(A) = \left\{ \begin{bmatrix} 
            2 \\
            3 \\
            1 \\
            0 \\
        \end{bmatrix} \right\} $
        Rank = 3
    \end{center}
    \item .\\
    \begin{center}
        $AN(A)=\Vec{0}$ \\
        $\begin{bmatrix}
            1 & 0 & -2 & 0 \\
            0 & 1 & -3 & 0 \\
            0 & 0 & 0 & 1 \\
        \end{bmatrix}
        \begin{bmatrix}
            2 \\ 3 \\ 2 \\ 0
        \end{bmatrix}
        = \Vec{0}$
    \end{center}
    \item All $\Vec{b}$ can be solved because A has full row rank and $C(A)$ spans all of $\mathbb{R}^3$
\end{enumerate}
	\end{solution}
	
	\question \S 8.1 \# 3 Determine whether each of the following is a linear transformation $\mathbb{R}^2 \to \mathbb{R}^2$ and explain.
	
	\begin{multicols}{2}
	\begin{enumerate}[label=(\alph*)]
	\item $T((v_1,v_2)^\intercal) = (v_2,v_1)^\intercal$
	\item $T((v_1,v_2)^\intercal) = (v_1,v_1)^\intercal$
	\item $T((v_1,v_2)^\intercal) = (0,v_1)^\intercal$
	\item $T((v_1,v_2)^\intercal) = (0,1)^\intercal$
	\item $T((v_1,v_2)^\intercal) = (v_1-v_2,v_1-v_2)^\intercal$
	\item $T((v_1,v_2)^\intercal) = (v_1v_2,0)^\intercal$
	\end{enumerate}
	\end{multicols}
	
	\begin{solution}
	$\begin{bmatrix} \begin{tabular}{cccc|c}
		1 & 3 & 1 & 2 & 1 \\
		2 & 6 & 4 & 8 & 3 \\
		0 & 0 & 2 & 4 & 1 \\
		\end{tabular} \end{bmatrix}
		\xrightarrow{R_2-2R_1}
		\begin{bmatrix} \begin{tabular}{cccc|c}
		1 & 3 & 1 & 2 & 1 \\
		0 & 0 & 2 & 4 & 1 \\
		0 & 0 & 2 & 4 & 1 \\
		\end{tabular} \end{bmatrix}
		\xrightarrow{R_3-R_2}
		\begin{bmatrix} \begin{tabular}{cccc|c}
		1 & 3 & 1 & 2 & 1 \\
		0 & 0 & 2 & 4 & 1 \\
		0 & 0 & 0 & 0 & 0 \\
		\end{tabular} \end{bmatrix}$.
		We only have 2 columns with pivots after elimination the remaining two columns are our free variables. We can now solve for the $N(A)$,
		\newline
	    $\begin{tabular}{ccccc}
		x+3y+z+2t=0 & Given & y=1 & then & x=-3 \\ 
		 & & t=0 &  & z=0 \\
		2z+4t=0 & Given & y=0 & then & x=1 \\ 
		 & & t=1 &  & z=-2 \end{tabular}$.
		 \newline
		 So $N(A)=Span \left( 
		 \begin{bmatrix} -3 \\ 1 \\ 0 \\ 0 \\ \end{bmatrix}, 
		 \begin{bmatrix} 1 \\ 0 \\ -2 \\ 1 \end{bmatrix}\right)$
		\newline
		We can now solve for $x_{particular}=\begin{bmatrix} 1/2 \\ 0 \\ 1/2 \\ 0 \end{bmatrix}$.  
		\newline
		The complete solution is,
		$\begin{bmatrix} 1/2 \\ 0 \\ 1/2 \\ 0 \end{bmatrix} + \alpha \begin{bmatrix} -3 \\ 1 \\ 0 \\ 0 \\ \end{bmatrix} + \beta \begin{bmatrix} 1 \\ 0 \\ -2 \\ 1 \end{bmatrix}$
		\newline
		Since $\alpha$ and $\beta$ can be any number we can have infinite solutions.
	\end{solution}
	
	\question The definition of a linear transformation does not say anything about the zero vector, yet I've mentioned that a linear transformation must send the zero vector in the domain to the zero vector in the codomain. Use the definition to show that $T(\vec{0}) = \vec{0}$ for any linear transformation $T$.
	
	\begin{solution}
	To tell if a transformation is linear you use the following definition. $T(\Vec{x})+T(\Vec{y})=T(\Vec{x}+\Vec{y})$ and $\alpha T(\Vec{x})=T(\alpha\Vec{x})$
\\
$\alpha T(\Vec{x}) = T(\alpha\Vec{x})$
\\
$x=0$
\\
$T(0\Vec{x})= T(\Vec{0})=\Vec{0}T(\Vec{x})=\Vec{0}T(\Vec{0})=0$
\\
$T(\Vec{0})+T(\Vec{0})=0+0=0$

	\end{solution}
	
	\question Find the matrix representation of each linear transformation.
	
	\begin{enumerate}[label=(\alph*)]
	
	Given, $C(A)=Span \left( 
	\begin{bmatrix}
	1 \\ 2 \\ 4
	\end{bmatrix}
	,
	\begin{bmatrix}
	2 \\ 2 \\ 1
	\end{bmatrix}
	\right)$
	and 
	$C(A^T)=Span \left( 
	\begin{bmatrix}
	1 \\ 0
	\end{bmatrix}
	,
	\begin{bmatrix}
	1 \\ 1
	\end{bmatrix}
	\right)$
	We can construct the outer product from the column space of $C(A)$ and $C(A^T)$. So,
	\newline
	$
	\begin{bmatrix}
	1 \\ 2 \\ 4
	\end{bmatrix}
	\begin{bmatrix}
	1 & 0
	\end{bmatrix}
	+
	\begin{bmatrix}
	2 \\ 2 \\ 1
	\end{bmatrix}
	\begin{bmatrix}
	1 & 1
	\end{bmatrix}
	=A=
	\begin{bmatrix}
	1 & 0 \\
	2 & 0 \\
	4 & 0
	\end{bmatrix}
	+
	\begin{bmatrix}
	2 & 2 \\
	2 & 2 \\
	1 & 1
	\end{bmatrix}
	=
	\begin{bmatrix}
	3 & 2 \\
	4 & 2 \\
	5 & 1
	\end{bmatrix}
	$
	\newline
	Using the outer product vectors we can create 2 matrices and verify that the product equals $A$.
	$
	\begin{bmatrix}
	1 & 2 \\
	2 & 2 \\
	4 & 1
	\end{bmatrix}
	\begin{bmatrix}
	1 & 0 \\
	1 & 1 \\
	\end{bmatrix}
	=
	\begin{bmatrix}
	3 & 2 \\
	4 & 2 \\
	5 & 1
	\end{bmatrix}
	=A
	$	
	\end{enumerate}
	
	\question \textbf{The Crazy Vector Space Returns.} Recall the ``crazy vector space'', which we will denote by $V$, from HW 2. As a reminder, the vectors in $V$ are $2 \times 1$ matrices, with vector addition and scalar multiplication defined by
	\[
	\begin{bmatrix}
	x_1 \\ x_2
	\end{bmatrix}
	\oplus
	\begin{bmatrix}
	y_1 \\ y_2
	\end{bmatrix}
	=
	\begin{bmatrix}
	x_1+y_1+1 \\ x_2+y_2+1
	\end{bmatrix},
	\qquad
	\alpha
	\odot
	\begin{bmatrix}
	x_1 \\ x_2
	\end{bmatrix}
	=
	\begin{bmatrix}
	\alpha x_1 + \alpha - 1 \\
	\alpha x_2 + \alpha - 1
	\end{bmatrix}.
	\]
	Also recall that the zero vector in $V$ is $\begin{smallbmatrix} -1 \\ -1 \end{smallbmatrix}$.
	
	\begin{enumerate}[label=(\alph*)]
    		\item Verify that 
		$\mathcal{B} = \left\lbrace
		\begin{bmatrix} 1 \\ -1 \end{bmatrix},
		\begin{bmatrix} -1 \\ 1 \end{bmatrix}\right\rbrace$ is a basis for $V$. 
		Is this at all counter-intuitive to you? (It's okay if the answer is no.)
		
		\begin{solution}
            $a \odot \begin{bmatrix} 1 \\ -1 \end{bmatrix} \oplus \beta \odot \begin{bmatrix} -1 \\ 1 \end{bmatrix}$
            \\
            $\begin{bmatrix} a + a-1 \\ -a + a -1 \end{bmatrix} \oplus 
            \begin{bmatrix} -\beta + \beta -1 \\ \beta + \beta -1 \end{bmatrix} = 
            \begin{bmatrix}2 \alpha-1 \\ 2 \beta -1 \end{bmatrix}$
            
            With $\alpha$ and $\beta$ you can span all of $R^2$
            \\
            Proof that the vectors are linearly independent.
            $c_1 \odot \begin{bmatrix} 1 \\ -1\end{bmatrix}
            \oplus c_2 \odot \begin{bmatrix} -1 \\ 1\end{bmatrix} =  
            \begin{bmatrix} -1 \\ -1\end{bmatrix}
            $
            \\
            $
             \begin{bmatrix} c_1+c_1-1 \\ -c_1+c_1-1\end{bmatrix} \oplus
             \begin{bmatrix} -c_2+c_2-1 \\ c_2+c_2-1\end{bmatrix}=  
             \begin{bmatrix} -1 \\ -1\end{bmatrix}
            $
            \\
            $
            \begin{bmatrix} 2c_1-1 \\ -1\end{bmatrix} \oplus
            \begin{bmatrix} -1 \\ 2c_2-1\end{bmatrix} =
            \begin{bmatrix} 2c_1-1 \\ 2c_2-1\end{bmatrix} =
            \begin{bmatrix} -1 \\ -1\end{bmatrix}
            $
            \\
            The only valid solution for this is $c_1=c_2=0$ which proves linear independence.
		\end{solution}
		
		\item Determine whether
		$\left\lbrace
		\begin{bmatrix} 1 \\ 2 \end{bmatrix},
		\begin{bmatrix} 3 \\ 5 \end{bmatrix}\right\rbrace$	
		is a basis for $V$.
		Is your conclusion at all counter-intuitive to you? (It's okay if the answer is no.)
		
		\begin{solution}
		    It is not a solution for $V$. \\
		    $\alpha\odot\begin{bmatrix}1\\2\end{bmatrix}\oplus\beta\odot\begin{bmatrix}3\\5\end{bmatrix}$
		    \\
		    $\begin{bmatrix}\alpha+\alpha-1 \\ 2\alpha+\alpha-1\end{bmatrix}\oplus
		    \begin{bmatrix}3\beta+\beta-1 \\ 5\beta+\beta-1\end{bmatrix} = 
		    \begin{bmatrix}2\alpha+4\beta-1 \\ 3\alpha+6\beta-1\end{bmatrix}$
		    \\
		    This does not span all of $\mathbb{R}^2$.  You could not reach $\begin{bmatrix}1000 \\ 100\end{bmatrix}$
		\end{solution}
		
		\item Let $T\colon V \to V$ be the function defined by
		\[
		T\left(\begin{bmatrix}a\\b\end{bmatrix}\right) =
		\begin{bmatrix} 2b+1 \\ 6a+5 \end{bmatrix}.
		\]
		
		Verify that $T$ is a linear transformation. (Hint: Be \textit{very} careful with the operations in $V$. Use $\oplus$ and $\odot$ to help distinguish the operations.)
		
		\begin{solution}
		    Validate $T(\Vec{x})+T(\Vec{y})=T(\Vec{x}+\Vec{y})$ and $\alpha T(\Vec{x}) = T(\alpha\Vec{x})$\\
		    $T\left(\begin{bmatrix}a \\b\end{bmatrix}\right)= \begin{bmatrix}2b+1 \\ 6a+5\end{bmatrix}$ \\
		    $T\left(\begin{bmatrix}x \\y\end{bmatrix}\right)= \begin{bmatrix}2y+1 \\ 6x+5\end{bmatrix}$
		$T\left(\begin{bmatrix}a \\b\end{bmatrix}\right) \oplus T \left(\begin{bmatrix}x \\y\end{bmatrix}\right)= \begin{bmatrix}2b+1 \\ 6a+5\end{bmatrix}\oplus\begin{bmatrix}2y+1 \\ 6x+5\end{bmatrix} = \begin{bmatrix}2b2y+3 \\ 6a+6x+11\end{bmatrix}$
		\\
		$\begin{bmatrix}a \\b\end{bmatrix}\oplus\begin{bmatrix}x \\y\end{bmatrix} = \begin{bmatrix}a+x+1\\b+y+1\end{bmatrix}$
		\\
		$T\left( \begin{bmatrix}a+x+1\\b+y+1\end{bmatrix}\right) = \begin{bmatrix}2b+2y+3\\6a+6x+11\end{bmatrix}$
		\\
		$\alpha T\begin{bmatrix}a\\b\end{bmatrix}=\alpha\odot\begin{bmatrix}2b+1\\6a+5\end{bmatrix}=\begin{bmatrix}2\alpha b+ 2\alpha-1\\6\alpha a + 6\alpha -1\end{bmatrix}$
		\\
		$\alpha\odot\begin{bmatrix}a \\ b\end{bmatrix}= \begin{bmatrix}\alpha a + \alpha -1\\ \alpha b +\alpha-1\end{bmatrix}$
		\\
		$T\left(\begin{bmatrix}\alpha a + \alpha -1\\ \alpha b +\alpha-1\end{bmatrix} \right)=\begin{bmatrix}2\alpha b+ 2\alpha-1\\6\alpha a + 6\alpha -1\end{bmatrix}$
		\end{solution}	
		
		\item Find the matrix representation for $T$ with respect to the basis $\mathcal{B}$.
		
		\begin{solution} \\
            $T\left( \begin{bmatrix}1 \\ -1 \end{bmatrix} \right) = \begin{bmatrix} -1 \\ 11 \end{bmatrix}$
            \\
            $T\left( \begin{bmatrix}-1 \\ 1 \end{bmatrix} \right) = \begin{bmatrix} 3 \\ -1 \end{bmatrix}$
            \\
            $\begin{bmatrix}-1\\11\end{bmatrix} = c_1\odot\begin{bmatrix}1\\-1\end{bmatrix}\oplus c_2\odot\begin{bmatrix}-1\\1\end{bmatrix}$
            \\
            $\begin{bmatrix}2c_1-1 \\ 2c_2-1\end{bmatrix} = \begin{bmatrix} -1 \\ 11 \end{bmatrix}$
            \\
            Therefore, 
            $c_1 = 0$ and $c_2=6$.
            \\
            $\begin{bmatrix} -1 \\ 11 \end{bmatrix}\to \begin{bmatrix} 0 \\ 6 \end{bmatrix}$
            \\
            $\begin{bmatrix} -3 \\ -1 \end{bmatrix}=c_1\odot\begin{bmatrix} 1 \\ -1 \end{bmatrix}\oplus c_2\begin{bmatrix} -1 \\ 1 \end{bmatrix}$
            \\
            $\begin{bmatrix} 2c_1-1 \\ 2c_2-1 \end{bmatrix}= \begin{bmatrix} 3 \\ -1 \end{bmatrix}$
            \\
            Therefore, $c_1 = 2$  and $c_2=0$
            \\
            $\begin{bmatrix} 3 \\ -1 \end{bmatrix}\to\begin{bmatrix} 2 \\ 0 \end{bmatrix}$
            \\
            $A=\begin{bmatrix} 0 & 2 \\ 6 & 0 \end{bmatrix}$
		\end{solution}
		
	\end{enumerate}
	
\end{questions}





\end{document}