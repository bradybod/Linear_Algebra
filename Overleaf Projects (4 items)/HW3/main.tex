\documentclass[]{exam}

\usepackage{amsmath, amssymb}
\usepackage[margin=1in]{geometry}
\usepackage[en-US, showdow]{datetime2}
\usepackage{graphicx}

\usepackage{arydshln}

\newcommand*\circled[1]{\tikz[baseline=(char.base)]{
            \node[shape=circle,draw,inner sep=2pt] (char) {#1};}}


\newenvironment{smallbmatrix}
{\left[\begin{smallmatrix}}
{\end{smallmatrix}\right]}


\title{Homework 3\\
	Due \DTMdate{2020-02-21} %Add a % at the begining of the line to remove the due date from the title (or just delete this line entirely) 
	}
\date{ %\today %You can remove the first % in this line to show the current days date.
	}
\author{Name: Brady Bodily \\
		\footnotesize Collaborators: Mitch Pound  \\
		\footnotesize Recitation Section: 504/505/506
		%You should probably change this to your actual name, at least if you want credit.
		}
\begin{document}
\maketitle

\printanswers %comment out this line to hide your answers.

\begin{questions}
	\question Let $L$ be an invertible matrix such that $A = LU$. Prove that $U$ and $A$ have the same nullspace. Your proof must consist of two parts. \\
	1) Show that if $U\vec{x}=\vec{0}$, then $A\vec{x}=\vec{0}$.
	2) Show that if $A\vec{x}=\vec{0}$, then $U\vec{x}=\vec{0}$.
	
	\begin{solution}
	    \begin{center}
\begin{tabular}{ c | c c c c c}
\hline
 Rank & 0& 1 & 2 & 3 & 4 \\ 
 \hline
 $dim(C(A))$ & 0 & 1 & 2  & 3 & 4\\  
 $dim(N(A))$ & 4 & 3 & 2 & 1 & 0 \\
 $dim(C(A^T))$ & 0 & 1 & 2 & 3 & 4 \\
 $dim(N(A^T))$ & 6 & 5 & 4 & 3 & 2\\ 
 
\end{tabular}
\end{center}
	\end{solution}
	
	\question If $AB=C$, the rows of $C$ are combinations of the rows of which matrix? What information does this relationship give you about the rank of $C$? Now consider $B^\intercal A^\intercal= C^\intercal$. What can you say about the rank of $C^\intercal$? Finally, since the rank of $C$ and $C^\intercal$ are the same, what can you say about the rank of $C$ based on the ranks of $A$ and $B$? 
	
	\begin{solution}
	    Using the row method of matrix multiplication we can easily see which matrices rows are combinations of $C$.  Given the matrices $AB=C \Rightarrow$ {$\begin{bmatrix} \begin{tabular}{cc|cc:c}
		     a & b & e & f & a(e,f)+b(e,f)  \\
		     c & d & g & h & c(g,h)+d(g,h) \\
		\end{tabular} \end{bmatrix}$}, therefore rows of $C$ are a linear combination of the matrix $B$'s rows with coefficients from $A$ resulting in  \begin{center} $R(C) \le R(B)$.\end{center}  Given this we now know that for $B^TA^T= C^T$ the rows of $C^T$ are linear combinations of  rows of $A^T$ and as before $R(C^T)\le R(A^T)$. Since $R(A^T)=R(A)$ we can state $R(C)=min\{R(A),R(B)\}$ Given pivots remain in the same columns and rows during a transformation we know $R(C)=R(C^T)$.  Finally given $R(A)$, $R(B)$, $A$ is $m\times n$, $B$ is $n\times m$, and the matrices are are inversible. We now know \begin{center} $R(C)=min\{R(A),R(B)\}$, \end{center} it is also interesting to note that $R(A)+R(B)-n=min\{R(A),R(B)\}$ so we can also state $R(C)$ is calculated by $R(A)+R(B)-n=R(C)$.

	\end{solution}
	
	\question Suppose that $A$ is an $m \times n$ matrix and $B$ is a $p \times n$ matrix. We can build the so-called ``block matrix'' $C = \begin{bmatrix} A \\ B \end{bmatrix}$. How is the nullspace $N(C)$ related to the spaces $N(A)$ and $N(B)$?\footnote{For instance, is $N(C)=N(A)$? Is $N(C) \subset N(A)$? Look for relationships such as these, possibly involving all three: $A$, $B$, and $C$.} Now, suppose that $A$ is an invertible 4 by 4 matrix. Give a basis for the nullspace of $D = \begin{bmatrix} A & A \end{bmatrix}$. 
	
	\begin{solution}
	    
\begin{enumerate}[label=(\alph*)]
    \item .\\
    \begin{center}
        $A\Vec{x} = \Vec{0}$ \\
        $A \Vec{x_n}+A \Vec{x_p} = \Vec{b} + \Vec{0}$ \\
        $A(\Vec{x_n} + \Vec{x_p})=\Vec{b}$ \\
        $A\left( \alpha 
        \begin{bmatrix} 
            2 \\
            3 \\
            1 \\
            0 \\
        \end{bmatrix} \right) = \Vec{0}$
        
        $N(A) = \left\{ \begin{bmatrix} 
            2 \\
            3 \\
            1 \\
            0 \\
        \end{bmatrix} \right\} $
        Rank = 3
    \end{center}
    \item .\\
    \begin{center}
        $AN(A)=\Vec{0}$ \\
        $\begin{bmatrix}
            1 & 0 & -2 & 0 \\
            0 & 1 & -3 & 0 \\
            0 & 0 & 0 & 1 \\
        \end{bmatrix}
        \begin{bmatrix}
            2 \\ 3 \\ 2 \\ 0
        \end{bmatrix}
        = \Vec{0}$
    \end{center}
    \item All $\Vec{b}$ can be solved because A has full row rank and $C(A)$ spans all of $\mathbb{R}^3$
\end{enumerate}
	\end{solution}
	
	\question Consider the following system. Show the system has an infinite number of solutions. Then give the ``complete solution''. 
	
	\[\begin{bmatrix}
	1 & 3 & 1 & 2\\
	2 & 6 & 4 & 8\\
	0 & 0 & 2 & 4\\
	\end{bmatrix}
	\begin{bmatrix}
	x \\ y \\ z \\ t
	\end{bmatrix} = 
	\begin{bmatrix}
	1 \\ 3 \\ 1
	\end{bmatrix}\]
	\begin{solution}
	    $\begin{bmatrix} \begin{tabular}{cccc|c}
		1 & 3 & 1 & 2 & 1 \\
		2 & 6 & 4 & 8 & 3 \\
		0 & 0 & 2 & 4 & 1 \\
		\end{tabular} \end{bmatrix}
		\xrightarrow{R_2-2R_1}
		\begin{bmatrix} \begin{tabular}{cccc|c}
		1 & 3 & 1 & 2 & 1 \\
		0 & 0 & 2 & 4 & 1 \\
		0 & 0 & 2 & 4 & 1 \\
		\end{tabular} \end{bmatrix}
		\xrightarrow{R_3-R_2}
		\begin{bmatrix} \begin{tabular}{cccc|c}
		1 & 3 & 1 & 2 & 1 \\
		0 & 0 & 2 & 4 & 1 \\
		0 & 0 & 0 & 0 & 0 \\
		\end{tabular} \end{bmatrix}$.
		We only have 2 columns with pivots after elimination the remaining two columns are our free variables. We can now solve for the $N(A)$,
		\newline
	    $\begin{tabular}{ccccc}
		x+3y+z+2t=0 & Given & y=1 & then & x=-3 \\ 
		 & & t=0 &  & z=0 \\
		2z+4t=0 & Given & y=0 & then & x=1 \\ 
		 & & t=1 &  & z=-2 \end{tabular}$.
		 \newline
		 So $N(A)=Span \left( 
		 \begin{bmatrix} -3 \\ 1 \\ 0 \\ 0 \\ \end{bmatrix}, 
		 \begin{bmatrix} 1 \\ 0 \\ -2 \\ 1 \end{bmatrix}\right)$
		\newline
		We can now solve for $x_{particular}=\begin{bmatrix} 1/2 \\ 0 \\ 1/2 \\ 0 \end{bmatrix}$.  
		\newline
		The complete solution is,
		$\begin{bmatrix} 1/2 \\ 0 \\ 1/2 \\ 0 \end{bmatrix} + \alpha \begin{bmatrix} -3 \\ 1 \\ 0 \\ 0 \\ \end{bmatrix} + \beta \begin{bmatrix} 1 \\ 0 \\ -2 \\ 1 \end{bmatrix}$
		\newline
		Since $\alpha$ and $\beta$ can be any number we can have infinite solutions.
	\end{solution}
	
	\question \S 3.4 \# 26 Recall that $\mathcal{M}_3$ is the vector space of all $3 \times 3$ matrices with real entries. Prove that each of the following collections of matrices are a subpsace of $\mathcal{M}_3$. Find a basis for each of subspaces of 3 by 3 matrices, then state the dimension of the subspace. Make sure to justify why your basis is, in fact, a basis.
	
	\begin{parts}
		\part All diagonal matrices.
			\begin{solution}
	            	$\begin{bmatrix} 
	            1 & 0 & 0 \\
	            0 & 1 & 0 \\
	            0 & 0 & 1
	            \end{bmatrix} = 
	            \begin{bmatrix} 
	            1 & 0 & 0 \\
	            0 & 0 & 0 \\
	            0 & 0 & 0
	            \end{bmatrix}
	            +
	            \begin{bmatrix} 
	            0 & 0 & 0 \\
	            0 & 1 & 0 \\
	            0 & 0 & 0
	            \end{bmatrix}
	            +
	            \begin{bmatrix} 
	            0 & 0 & 0 \\
	            0 & 0 & 0 \\
	            0 & 0 & 1
	            \end{bmatrix}$
	            \newline
	            Since the matrices are linearly independent they can form a Basis. So the span of the matrices is,
	            \newline
	            $Span \left(
	            \begin{bmatrix} 
	            1 & 0 & 0 \\
	            0 & 0 & 0 \\
	            0 & 0 & 0
	            \end{bmatrix},
	            \begin{bmatrix} 
	            0 & 0 & 0 \\
	            0 & 1 & 0 \\
	            0 & 0 & 0
	            \end{bmatrix},
	            \begin{bmatrix} 
	            0 & 0 & 0 \\
	            0 & 0 & 0 \\
	            0 & 0 & 1
	            \end{bmatrix}
	            \right)$
	            Which tells us the dimension is 3. Any linear combination of these would result in a diagonal matrix.
			\end{solution}
		\part All symmetric matrices\footnote{A matrix $A$ is called \emph{symmetric} if $A^\intercal = A$.}
			\begin{solution}
	             We will use the matrix $A=\begin{bmatrix}
                a & b & c \\
                b & d & e \\
                c & e & f
                \end{bmatrix}$ which meets $A^T=A$ 
                \newline
                $\begin{bmatrix}
                a & b & c \\
                b & d & e \\
                c & e & f
                \end{bmatrix} = 
                \begin{bmatrix}
                a & 0 & 0 \\
                0 & 0 & 0 \\
                0 & 0 & 0
                \end{bmatrix}
                +
                \begin{bmatrix}
                0 & b & 0 \\
                b & 0 & 0 \\
                0 & 0 & 0
                \end{bmatrix}
                +
                \begin{bmatrix}
                0 & 0 & c \\
                0 & 0 & 0 \\
                c & 0 & 0
                \end{bmatrix}
                +
                \begin{bmatrix}
                0 & 0 & 0 \\
                0 & d & 0 \\
                0 & 0 & 0
                \end{bmatrix}
                +
                \begin{bmatrix}
                0 & 0 & 0 \\
                0 & 0 & e \\
                0 & e & 0
                \end{bmatrix}
                +
                \begin{bmatrix}
                0 & 0 & 0 \\
                0 & 0 & 0 \\
                0 & 0 & f
                \end{bmatrix}
               $
               Since they are linearly independent they can form a basis.
               \newline
                $Span = \left( \begin{bmatrix}
                a & 0 & 0 \\
                0 & 0 & 0 \\
                0 & 0 & 0
                \end{bmatrix}
                ,
                \begin{bmatrix}
                0 & b & 0 \\
                b & 0 & 0 \\
                0 & 0 & 0
                \end{bmatrix}
                ,
                \begin{bmatrix}
                0 & 0 & c \\
                0 & 0 & 0 \\
                c & 0 & 0
                \end{bmatrix}
                ,
                \begin{bmatrix}
                0 & 0 & 0 \\
                0 & d & 0 \\
                0 & 0 & 0
                \end{bmatrix}
                ,
                \begin{bmatrix}
                0 & 0 & 0 \\
                0 & 0 & e \\
                0 & e & 0
                \end{bmatrix}
                ,
                \begin{bmatrix}
                0 & 0 & 0 \\
                0 & 0 & 0 \\
                0 & 0 & f
                \end{bmatrix} \right)$
                $A=A^T$ for each of the matrices in the span. Since there are 6 matrices in the span the dimension is 6. Any linear combination of these would result in a symmetric matrix.
			\end{solution}
		\part All skew-symmetric matrices \footnote{A matrix $A$ is called \emph{skew-symmetric} if $A^\intercal = -A$.}
			\begin{solution}
	            \newline
		Using 1a and 1b we get the following.
		\newline
		\begin{center}
		    $(A+KI)\vec{x}=A\vec{x}+KI\vec{x}$
		    \newline
		    $\lambda \vec{x}+K\vec{x}=(\lambda+K)\vec{x}$
		    \newline
		\end{center}
		Therefore proving $\lambda + k$ is an eigenvalue of $A + kI$.

			\end{solution}
	\end{parts}
	
	\question In this exercise, we will explore the outer-product perspective of matrix multiplication. Every matrix whose rank is $r$ can be written as the sum of $r$ matrices of rank 1. An easy way to write a rank 1 matrix is using an outer-product (recall: $\vec{u}\vec{v}^\intercal$ is an outer-product). Construct a matrix $A$ with rank 2 that has $C(A) = \mathop{\mathbf{span}}((1,2,4)^\intercal, (2,2,1)^\intercal)$ and $C(A^\intercal) = \mathop{\mathbf{span}}((1,0)^\intercal,(1,1)^\intercal)$, you should use outer-products to find $A$.
	
\begin{solution}
    	Given, $C(A)=Span \left( 
	\begin{bmatrix}
	1 \\ 2 \\ 4
	\end{bmatrix}
	,
	\begin{bmatrix}
	2 \\ 2 \\ 1
	\end{bmatrix}
	\right)$
	and 
	$C(A^T)=Span \left( 
	\begin{bmatrix}
	1 \\ 0
	\end{bmatrix}
	,
	\begin{bmatrix}
	1 \\ 1
	\end{bmatrix}
	\right)$
	We can construct the outer product from the column space of $C(A)$ and $C(A^T)$. So,
	\newline
	$
	\begin{bmatrix}
	1 \\ 2 \\ 4
	\end{bmatrix}
	\begin{bmatrix}
	1 & 0
	\end{bmatrix}
	+
	\begin{bmatrix}
	2 \\ 2 \\ 1
	\end{bmatrix}
	\begin{bmatrix}
	1 & 1
	\end{bmatrix}
	=A=
	\begin{bmatrix}
	1 & 0 \\
	2 & 0 \\
	4 & 0
	\end{bmatrix}
	+
	\begin{bmatrix}
	2 & 2 \\
	2 & 2 \\
	1 & 1
	\end{bmatrix}
	=
	\begin{bmatrix}
	3 & 2 \\
	4 & 2 \\
	5 & 1
	\end{bmatrix}
	$
	\newline
	Using the outer product vectors we can create 2 matrices and verify that the product equals $A$.
	$
	\begin{bmatrix}
	1 & 2 \\
	2 & 2 \\
	4 & 1
	\end{bmatrix}
	\begin{bmatrix}
	1 & 0 \\
	1 & 1 \\
	\end{bmatrix}
	=
	\begin{bmatrix}
	3 & 2 \\
	4 & 2 \\
	5 & 1
	\end{bmatrix}
	=A
	$
\end{solution}
	
	\question Suppose that $A$ is a $m \times n$ matrix and $\vec{b}$ is a $m \times 1$ vector. Let $B$ be the $m \times (n+1)$ matrix formed by concatenating $\vec{b}$ to $A$, so $B = [A \, \vec{b}]$. What must be true so that $C(A)=C(B)$? What must be true if $C(A) \subset C(B)$ and $C(A) \ne C(B)$? Explain what must be true for $A \vec{x} = \vec{b}$ and $B \vec{x} = \vec{b}$ to have solutions. 
	
	
	\begin{solution}
        		\item Verify that 
		$\mathcal{B} = \left\lbrace
		\begin{bmatrix} 1 \\ -1 \end{bmatrix},
		\begin{bmatrix} -1 \\ 1 \end{bmatrix}\right\rbrace$ is a basis for $V$. 
		Is this at all counter-intuitive to you? (It's okay if the answer is no.)
		
		\begin{solution}
            $a \odot \begin{bmatrix} 1 \\ -1 \end{bmatrix} \oplus \beta \odot \begin{bmatrix} -1 \\ 1 \end{bmatrix}$
            \\
            $\begin{bmatrix} a + a-1 \\ -a + a -1 \end{bmatrix} \oplus 
            \begin{bmatrix} -\beta + \beta -1 \\ \beta + \beta -1 \end{bmatrix} = 
            \begin{bmatrix}2 \alpha-1 \\ 2 \beta -1 \end{bmatrix}$
            
            With $\alpha$ and $\beta$ you can span all of $R^2$
            \\
            Proof that the vectors are linearly independent.
            $c_1 \odot \begin{bmatrix} 1 \\ -1\end{bmatrix}
            \oplus c_2 \odot \begin{bmatrix} -1 \\ 1\end{bmatrix} =  
            \begin{bmatrix} -1 \\ -1\end{bmatrix}
            $
            \\
            $
             \begin{bmatrix} c_1+c_1-1 \\ -c_1+c_1-1\end{bmatrix} \oplus
             \begin{bmatrix} -c_2+c_2-1 \\ c_2+c_2-1\end{bmatrix}=  
             \begin{bmatrix} -1 \\ -1\end{bmatrix}
            $
            \\
            $
            \begin{bmatrix} 2c_1-1 \\ -1\end{bmatrix} \oplus
            \begin{bmatrix} -1 \\ 2c_2-1\end{bmatrix} =
            \begin{bmatrix} 2c_1-1 \\ 2c_2-1\end{bmatrix} =
            \begin{bmatrix} -1 \\ -1\end{bmatrix}
            $
            \\
            The only valid solution for this is $c_1=c_2=0$ which proves linear independence.
		\end{solution}
		
		\item Determine whether
		$\left\lbrace
		\begin{bmatrix} 1 \\ 2 \end{bmatrix},
		\begin{bmatrix} 3 \\ 5 \end{bmatrix}\right\rbrace$	
		is a basis for $V$.
		Is your conclusion at all counter-intuitive to you? (It's okay if the answer is no.)
		
		\begin{solution}
		    It is not a solution for $V$. \\
		    $\alpha\odot\begin{bmatrix}1\\2\end{bmatrix}\oplus\beta\odot\begin{bmatrix}3\\5\end{bmatrix}$
		    \\
		    $\begin{bmatrix}\alpha+\alpha-1 \\ 2\alpha+\alpha-1\end{bmatrix}\oplus
		    \begin{bmatrix}3\beta+\beta-1 \\ 5\beta+\beta-1\end{bmatrix} = 
		    \begin{bmatrix}2\alpha+4\beta-1 \\ 3\alpha+6\beta-1\end{bmatrix}$
		    \\
		    This does not span all of $\mathbb{R}^2$.  You could not reach $\begin{bmatrix}1000 \\ 100\end{bmatrix}$
		\end{solution}
		
		\item Let $T\colon V \to V$ be the function defined by
		\[
		T\left(\begin{bmatrix}a\\b\end{bmatrix}\right) =
		\begin{bmatrix} 2b+1 \\ 6a+5 \end{bmatrix}.
		\]
		
		Verify that $T$ is a linear transformation. (Hint: Be \textit{very} careful with the operations in $V$. Use $\oplus$ and $\odot$ to help distinguish the operations.)
		
		\begin{solution}
		    Validate $T(\Vec{x})+T(\Vec{y})=T(\Vec{x}+\Vec{y})$ and $\alpha T(\Vec{x}) = T(\alpha\Vec{x})$\\
		    $T\left(\begin{bmatrix}a \\b\end{bmatrix}\right)= \begin{bmatrix}2b+1 \\ 6a+5\end{bmatrix}$ \\
		    $T\left(\begin{bmatrix}x \\y\end{bmatrix}\right)= \begin{bmatrix}2y+1 \\ 6x+5\end{bmatrix}$
		$T\left(\begin{bmatrix}a \\b\end{bmatrix}\right) \oplus T \left(\begin{bmatrix}x \\y\end{bmatrix}\right)= \begin{bmatrix}2b+1 \\ 6a+5\end{bmatrix}\oplus\begin{bmatrix}2y+1 \\ 6x+5\end{bmatrix} = \begin{bmatrix}2b2y+3 \\ 6a+6x+11\end{bmatrix}$
		\\
		$\begin{bmatrix}a \\b\end{bmatrix}\oplus\begin{bmatrix}x \\y\end{bmatrix} = \begin{bmatrix}a+x+1\\b+y+1\end{bmatrix}$
		\\
		$T\left( \begin{bmatrix}a+x+1\\b+y+1\end{bmatrix}\right) = \begin{bmatrix}2b+2y+3\\6a+6x+11\end{bmatrix}$
		\\
		$\alpha T\begin{bmatrix}a\\b\end{bmatrix}=\alpha\odot\begin{bmatrix}2b+1\\6a+5\end{bmatrix}=\begin{bmatrix}2\alpha b+ 2\alpha-1\\6\alpha a + 6\alpha -1\end{bmatrix}$
		\\
		$\alpha\odot\begin{bmatrix}a \\ b\end{bmatrix}= \begin{bmatrix}\alpha a + \alpha -1\\ \alpha b +\alpha-1\end{bmatrix}$
		\\
		$T\left(\begin{bmatrix}\alpha a + \alpha -1\\ \alpha b +\alpha-1\end{bmatrix} \right)=\begin{bmatrix}2\alpha b+ 2\alpha-1\\6\alpha a + 6\alpha -1\end{bmatrix}$
		\end{solution}	
		
		\item Find the matrix representation for $T$ with respect to the basis $\mathcal{B}$.
		
		\begin{solution} \\
            $T\left( \begin{bmatrix}1 \\ -1 \end{bmatrix} \right) = \begin{bmatrix} -1 \\ 11 \end{bmatrix}$
            \\
            $T\left( \begin{bmatrix}-1 \\ 1 \end{bmatrix} \right) = \begin{bmatrix} 3 \\ -1 \end{bmatrix}$
            \\
            $\begin{bmatrix}-1\\11\end{bmatrix} = c_1\odot\begin{bmatrix}1\\-1\end{bmatrix}\oplus c_2\odot\begin{bmatrix}-1\\1\end{bmatrix}$
            \\
            $\begin{bmatrix}2c_1-1 \\ 2c_2-1\end{bmatrix} = \begin{bmatrix} -1 \\ 11 \end{bmatrix}$
            \\
            Therefore, 
            $c_1 = 0$ and $c_2=6$.
            \\
            $\begin{bmatrix} -1 \\ 11 \end{bmatrix}\to \begin{bmatrix} 0 \\ 6 \end{bmatrix}$
            \\
            $\begin{bmatrix} -3 \\ -1 \end{bmatrix}=c_1\odot\begin{bmatrix} 1 \\ -1 \end{bmatrix}\oplus c_2\begin{bmatrix} -1 \\ 1 \end{bmatrix}$
            \\
            $\begin{bmatrix} 2c_1-1 \\ 2c_2-1 \end{bmatrix}= \begin{bmatrix} 3 \\ -1 \end{bmatrix}$
            \\
            Therefore, $c_1 = 2$  and $c_2=0$
            \\
            $\begin{bmatrix} 3 \\ -1 \end{bmatrix}\to\begin{bmatrix} 2 \\ 0 \end{bmatrix}$
            \\
            $A=\begin{bmatrix} 0 & 2 \\ 6 & 0 \end{bmatrix}$
		\end{solution}
	\end{solution}
	
\end{questions}





\end{document}